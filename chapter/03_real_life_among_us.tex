\section{Real Life Among Us}
In den folgenden Abschnitten wird ein Konzept zum Spielen von Among Us als
interaktives Gesellschaftsspiel vorgestellt. Alle im vorherigen Kapitel
genannten Regeln und Prinzipen gelten weiterhin.
\subsection{Spielbereich}
\subsubsection{Allgemeines Konzept}
Der Spielbereich muss vor jeder Partie klar abgegrenzt werden. Hierzu eignet
sich beispielsweise ein großes Haus. Weitläufige leicht einsehbare Flächen
sollten vermieden werden.
\newline
Auf dem Spielbereich muss eine Meeting-Area definiert werden. In dieser finden
alle Abstimmungen statt. Die Meeting-Area dient auch als Startpunkt für jede
Hauptspielphase.
\newline
Auf dem Spielbereich verteilt befinden sich die einzelnen Aufgaben. Diese
Minigames sollten möglichst gleichmäßig verteilt sein, um die Spieler dazu
anzuregen sich über den gesamten Spielbereich hinweg zu bewegen.
\newline
Des Weiteren werden im Spielbereich Geräte installiert werden, um es den
Crewmates zu ermöglichen, den Spielbereich zu überwachen.

\subsubsection{Überwachungskameras}
Mithilfe mehrerer Smartphones und einem Tablet oder Laptop kann ein
Überwachungskamerasystem installiert werden. Die Smartphones filmen dabei
verschiedene Bereich und befinden sich mit dem Tablet oder Laptop in einer
Videokonferenz. Das überwachende Gerät sollte sich dabei nicht in der Nähe
der Meeting-Area befinden.

\subsubsection{Vitalmonitor}
Mithilfe der für die Kommunikation notwendigen Discord-Konferenz und einem
weiteren Tablet oder Laptop, kann ein Vitalmonitor eingerichtet werden, mit dem
der Lebendigkeitsstatus aller Spieler überwacht werden kann. Als
Lebendigkeitsstatus wird dazu der Sound-Mute-Button in Discord genutzt.
Wenn ein Spieler getötet wurde, muss er seinen Sound in der Discord-Konferenz
stummschalten.
Somit signalisieren die durchgestrichenen Kopfhörer neben dem Namen des Spielers
auf dem überwachenden Gerät, dass dieser getötet wurde.

\subsection{Kommunikation}
\subsubsection{Rollenvergabe}
Am Anfang jeder Partie erhält jeder einen Zettel, auf dem sich eine Rolle
befindet. Damit sich die Imposter gegenseitig identifizieren können, schließen
alle Spieler die Augen, damit anschließend nur die Imposter ihre wieder öffnen,
um sich ausfindig zu machen.
\subsubsection{Kommunikation während der Hauptspielphasen}
Während der Hauptspielphasen darf unter keinen Umständen untereinander
kommuniziert werden. Das gilt nicht nur für für verbale Kommunikation,
sondern auch für alle anderen Kommunikationswege, wie beispielsweise
Zeichensprache.
\subsubsection{Discord-Konferenz}
Um alle Spieler gleichzeitig zu einer Abstimmung rufen zu können, muss ein
direkter Kommunikationsweg zwischen den Spielern geschaffen werden.
Alle Spieler begeben sich mit ihren Smartphones in eine Discord-Konferenz.
Solange die Spieler am Leben sind, müssen sie ihre Mikrofon stummschalten und
ihre Audioausgabe aktiv haben.

\subsubsection{Kill-Reporting}
Wenn ein Spieler einen getöteten Spieler auffindet, kann er einen
``Kill reporten''. Dazu hebt er seine Stummschaltung in der Konferenz auf und
kündigt einen Kill-Report an. Anschließend begeben sich alle Spieler sofort in
die Meeting-Area und es beginnt eine Abstimmung.

\subsubsection{Emergency-Meeting}
Jeder Spieler darf einmal pro Partie ein ``Emergency Meeting'' einberufen. Dazu
hebt er, wie beim Kill-Reporting, seine Stummschaltung auf und kündigt es an.
Allerdings muss er sich dazu innerhalb der Meeting-Area befinden.
Anschließend begeben sich alle Spieler sofort in
die Meeting-Area und es beginnt eine Abstimmung.

\subsection{Spielmechaniken}
\subsubsection{Bewegung}
Jeder Spieler darf sich frei innerhalb des Spielbereichs bewegen. Allerdings
ist die Geschwindigkeit, mit der sich in Spieler bewegen kann, begrenzt. Man
darf nur gehen. Das bedeutet, dass zu jedem Zeitpunkt ein Fuß auf dem Boden sein
muss.

\subsubsection{Sichtfeld}
Das Sichtfeld jedes Spielers ist, wie im Online-Spiel, begrenzt, um das
Spielgeschehen spannend zu halten. Dazu werden Kappen genutzt, die so
tief zu tragen sind, dass sie das horizontale Sichtfeld stark einschränken. Bei
Dunkelheit können zusätzlich noch Handytaschenlampen hinzugezogen werden.

\subsubsection{Killing}
Imposter besitzen die Fähigkeit Crewmates zu töten. Um dies zu tun, muss dabei
eine Vorraussetzungen gegeben sein. Der Kill-Cooldown des Imposters muss
abgelaufen sein. Dieser wird zum Start jeder Hauptspielphase und nach jedem Kill
auf einen vorher vereinbarten Wert zurückgesetzt. Der Imposter ist dabei selbst
für das Zählen verantwortlich. Der Kill-Cooldowns beträgt 40s, kann aber
variiert werden.
\newline
Um einen Spieler zu töten, legt er seine Hand auf den Rücken oder die Schulter
eines Crewmates. Dieser muss sich anschließend unmittelbar auf den Boden legen.
Des Weiteren muss er sein Audioausgabe stummschalten, um dem Vitalmonitor seinen
Tod mitzuteilen.
\newline
Ein getöteter Crewmate darf erst nach dem nächsten Kill-Report oder Einberufung
eines Emergency-Meetings aufstehen und als Geist weiterspielen.

\subsubsection{Abstimmungen}
Jede Abstimmung beginnt mit einer Diskussion, die eine Minute dauert. Die Dauer
kann variiert werden.
Das anschließende Voting wird per Fingerzeig durchgeführt. Auf
einen 1,2,3-Countdown zeigen die Spieler auf den Spieler, den sie
rauswerfen wollen. Wer das Voting überspringen möchte, zeigt mit seinem Finger
auf den Boden. Nach dem Countdown darf keiner seine Auswahl ändern, bis das
Ergebnis des Votings geklärt wurde.
\newline
Rausgeworfene Spieler werden zu Geistern.

\subsubsection{Spielen als Geist}
Als Geist darf man nicht mehr aktiv am Spielgeschehen teilnehmen. Allerdings
kann man weiterhin seine Aufgaben absolvieren. Um zu kennzeichnen, dass es sich
bei einem getöteten Spieler um einen Geist handelt, tragen Geister keine Kappen.
Je nach Möglichkeit sind allerdings auch andere optische Identifikationskonzepte
denkbar.

\subsubsection{Sabotage}
Der aktuelle Stand des Konzepts enthält keine Konkreten Vorschläge für
Sabotageakte.

\subsubsection{Abkürzungen}
Imposter besitzen die Fähigkeit spezielle Abkürzungen innerhalb des
Spielbereichs nutzen zu dürfen. Da die Orte, an denen Real Life Among Us
gespielt werden kann, verschieden sind, muss vor jeder Partie festgelegt werden,
ob es Möglichkeiten für Abkürzungen gibt und wie diese von Impostern verwendet
werden können.


\subsection{Aufgaben}
Der aktuelle Stand des Konzepts enthält keine konkreten Vorschläge für Aufgaben.
Allerdings wird im Folgenden ein allgemeines Konzept für diese vorgestellt.
\newline
Jeder Crewmate erhält am Anfang jeder Partie eine Liste an Aufgaben, die sie
absolvieren müssen. Die Liste besteht aus einem kleinen Blatt Papier, auf dem
die Aufgaben stehen.  Die Imposter erhalten eine falsche Liste.
Jeder Spieler erhält einen kleinen Stift, mit dem er die absolvierten Aufgaben
abhaken kann. Nach dem Absolvieren aller Aufgaben, kann er den Zettel in der
Meeting-Area auf den Tisch legen.
\newline
Es wird zwischen drei verschiedenen Tasks
unterschieden. Sie unterscheiden sich in ihrem Schwierigkeitsgrad und der
benötigten Zeit.
\newline
Leichte Aufgaben dauern ca. 10 Sekunden. Jeder Spieler sollte sie sehr einfach
bewältigen können. Sie finden an einem einzigen Ort statt.
\newline
Mittlere Aufgaben dauern ca. 10 Sekunden. Jeder Spieler sollte sie sehr einfach
bewältigen können. Sie finden an verschiedenen Orten statt und beinhalten
eine Interaktion zwischen den verschiedenen Orten, an denen sie stattfinden.
Dabei können bis zu drei verschiedene Orte miteinander verknüpft sein.
\newline
Schwere Aufgaben dauern ca. 30 Sekunden. Jeder Spieler sollte sie nur mit etwas
Mühe oder Geschick bewältigen können. Sie finden an einem einzigen Ort statt.
\newline
Jedes Crewmate erhält am Start der Partie drei leichte Aufgaben, zwei mittlere
Aufgaben und eine schwere Aufgabe.
\newline
Imposter dürfen die Aufgaben nie richtig erfüllen. Sie dürfen nur so tun, als
würden sie die Aufgabe absolvieren. Sie dürfen diese nicht erfolgreich
abschließen oder müssen bei der Durchführung von der eigentlichen Anweisung der
Aufgabe abweichen.
\newline
Die Beschreibung jeder einzelnen Aufgabe befindet sich im Anhang.


\subsection{Vertrauen}
Das gesamte Spielkonzept von Real Life Among Us basiert auf Vertrauen. An
mehreren Stellen können Spieler schummeln. Um die Möglichkeit des Schummelns zu
eliminieren, müsste man einen erheblichen Mehraufwand aufstellen, der die
Komplexität der Vorbereitung und Durchführung einer Partie stark erhöhen würde.
Daher wird bei Konzepten, wie dem Kill-Cooldown oder dem Abhaken von Aufgaben,
auf die Ehrlichkeit und Fairness der Spieler gesetzt.
