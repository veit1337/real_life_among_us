\section{Among Us}
In den folgenden Abschnitten werden die allgemeinen Prinzipien und Regeln von
Among Us erläutert.
\subsection{Spielkonzept}
Das Spielkonzept von Among Us ist vergleichbar mit bereits bekannten Spielen,
wie Mafia, Secret Hitler oder Werwölfe.
\newline
Es findet auf einer begrenzten
Spielkarte statt, auf der sich die vier bis 15 Spieler frei bewegen können.
Jeder Spieler erhält vor dem Start jeder Partie zufällig eine Rolle, Crewmate
oder Imposter. Je nach Spieleranzahl, kann es bis zu 3 Imposter geben.
Das Ziel der Imposter ist es, alle Crewmates zu eliminieren.
Das Ziel der Crewmates ist es, ihre Aufgaben, auf der Karte verteilte
Minispiele, zu absolvieren und die Imposter in Abstimmungen aus der Partie zu
werfen.
Ein getöteter oder rausgeworfener Spieler wird zu einem Geist.
Ein Geist kann nicht mehr aktiv am Spielgeschehen teilnehmen. Allerdings kann er
weiterhin seine Aufgaben erledigen.
\newline
Die Imposter gewinnen die Partie, sobald genauso viele Imposter wie Crewmates
am Leben sind oder ein tötlicher Sabotageakt erfolgreich war. Die Crewmates
gewinnen die Partie, wenn alle Crewmates und Geister alle ihre Aufgaben
absolviert haben oder
wenn kein Imposter mehr in der Partie ist. Die Rolle eines jeden Spielers ist
bis zum Ende der Partie unbekannt.

\subsection{Spielphasen}
\subsubsection{Hauptspielphase}
Während der Hauptspielphase können sich alle Spieler frei auf der Karte bewegen.
Die Spieler dürfen dabei in keinster Weise miteinander kommunizieren.
Crewmates und Geister können ihre Aufgaben erledigen. Imposter können Crewmates
töten. Um einen Crewmate zu töten, muss ein Imposter sich ohne Kill-Cooldown
innerhalb der Kill-Distanz befinden.
Des Weiteren können Imposter spezielle Abkürzungen benutzen, die nur
ihnen zugänglich sind, um sich auf der Karte schneller fortbewegen zu können.
Außerdem können sie Sabotagen starten, die von jedem Spieler rückgängig
gemacht werden können. Jeder Spieler kann die von Karte zu Karte
unterschiedlichen Geräte nutzen, wie beispielsweise Überwachungskameras.
\newline
Jede Partie beginnt mit einer Hauptspielphase. Eine Hauptspielphase kann durch
einen Kill-Report, ein Emergency-Meeting oder den Sieg der Imposter oder
Crewmates beendet werden.

\subsubsection{Abstimmung}
Eine Abstimmmung besteht aus zwei Teilen. Im ersten Teil, der Diskussion, können
alle Spieler miteinander kommunizieren. Im zweiten Teil, dem Voting, hat jeder
Spieler eine Stimme. Diese kann er dazu einsetzen, um einen Spieler aus der
Partie zu werfen oder das Voting zu überspringen. Wenn die Mehrheit der Spieler
für das Überspringen des Votings stimmt, wird die Abstimmung ohne weiteres
beendet. Wenn die Mehrheit der Spieler für den Rauswurf eines Spielers stimmt,
wird dieser aus der Partie geworfen. Bei einem Gleichstand in jeder Weise wird
die Abstimmung ohne weiteres beendet.
\newline
Eine Abstimmung kann durch einen Kill-Report oder ein Emergency-Meeting
ausgelöst werden. Falls nach einer Abstimmung die Partie nach keiner
gewonnen hat, folgt eine Hauptspielphase.
